\documentclass[a4paper]{article}
\usepackage{amsmath,amssymb,caption,float,graphicx,xcolor}
% \usepackage{minted}
% \usepackage[utf8]{inputenc}
% \usepackage[english]{babel}
% \usepackage[backend=bibtex]{biblatex}
% \addbibresource{Lab3.bib}
\captionsetup[figure]{labelsep=period}
% \definecolor{bg}{rgb}{0.95,0.95,0.95}
\renewcommand\thesection{\arabic{section}}
% \usemintedstyle{emacs}
\begin{document}
\begin{center}
    \huge
    \textbf{VE482\\Introduction to Operating Systems\\}
    \Large
    \vspace{15pt}
    \uppercase{\textbf{Homework 6}}\\
    \large
    \vspace{5pt}\today\\
    \vspace{5pt}
    Yihua Liu 518021910998
    \vspace{5pt}
    \rule[-5pt]{.97\linewidth}{0.05em}
\end{center}
\section*{Ex. 1 — Simple questions}
\begin{enumerate}
    \item Consider a swapping system in which memory consists of the following hole sizes in memory order: 10 KB, 4 KB, 20 KB, 18 KB, 7 KB, 9 KB, 12 KB, and 15 KB. Assuming first fit is used, which hole is taken for successive segment requests of: (i) 12 KB, (ii) 10 KB and (iii) 9KB. Repeat for best fit and quick fit.
\end{enumerate}
\end{document}